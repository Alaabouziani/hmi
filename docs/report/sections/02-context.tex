\section{Context}
\label{sec:context}

The case study considers a desktop application that handles both user registration and authentication. The registration form initially presents nine interactive elements:
\begin{itemize}
  \item Last Name (optional)
  \item First Name (optional)
  \item Username
  \item Email
  \item Phone Number
  \item Password
  \item Confirm Password
  \item Validate button
  \item Cancel button
\end{itemize}

The interface must clearly communicate validation errors after the user presses the \textit{Validate} button, and it must reassure the user when the submission succeeds by showing a summary screen with the captured information and two controls: a final \textit{Validate} button and a \textit{Back} button for corrections.

\subsection*{Field Constraints}

\begin{longtable}{@{}p{0.25\textwidth}p{0.68\textwidth}@{}}
\toprule
\textbf{Field} & \textbf{Description and Validation Rule} \\
\midrule
Last Name & Optional. When provided, it is displayed verbatim in the confirmation screen. \\
First Name & Optional. When provided, it is displayed verbatim in the confirmation screen. \\
Username & Required if both Last Name and First Name are empty. The value must be non-empty once that condition applies. \\
Email & Required if the Phone Number is empty. Must follow a valid email pattern (\texttt{local@domain}). \\
Phone Number & Required if the Email is empty. Must respect the application's digit and length constraints. \\
Password & Always required. Collected securely and never echoed back in plaintext. \\
Confirm Password & Always required and must match the Password field exactly. \\
Validate & Submits the form, triggering validation and either success confirmation or targeted feedback. \\
Cancel & Aborts the process and returns the user to the neutral state. \\
\bottomrule
\end{longtable}

\subsection*{Success Workflow}

When every constraint is satisfied, the user advances to a confirmation view summarising all captured inputs (excluding raw passwords). The user must explicitly confirm the submission or return to the editable form via the \textit{Back} button to adjust any field before final validation.

